\documentclass[a4paper, 12pt, oneside]{report}
\raggedbottom
\usepackage{setspace}
\onehalfspacing
\newcommand{\HRule}{\rule{\linewidth}{0.5mm}}
\pdfminorversion=6
\usepackage{amssymb}
\usepackage{enumitem}
\usepackage[french]{babel}
\usepackage{wasysym}
\usepackage[a4paper,bindingoffset=0in,left=1in,right=1in,top=1in,bottom=1in,footskip=.25in]{geometry}
\usepackage[ansinew]{inputenc}
\usepackage{color}% colour for table cells
\usepackage{multirow}
\definecolor{light-gray}{gray}{0.6}
\usepackage{rotating}
\usepackage{booktabs}
\usepackage{sidecap}
\usepackage{caption}
    \captionsetup{font=small,labelfont=bf}
\usepackage{floatrow}
\usepackage{array,multirow}
\floatsetup[table]{capposition=top}
\parindent 0pt	

\usepackage{hyperref}

\title{\vspace{-15mm}\fontsize{24pt}{10pt}\selectfont\textbf{Machine Learning For a House pricing Prediction Web Application}} % Article title

\author{
\large
\textsc{Alex Soudant}
\\[2mm] 
\normalsize Ynov Ing�sup M1,\\
\normalsize 20 Boulevard G�n�ral de Gaulle, 44200 Nantes \\
\\[2mm] 
\normalsize Correspondence: Alex Soudant. E-mail: \href{mailto:alex.soudant@ynov.com}{alex.soudant@ynov.com} % Your email address
\vspace{-5mm}
}

\date{}

\vspace{1cm}

\begin{document}

\maketitle

\vspace*{5cm}
\textbf{Abstract}

\noindent
\normalfont 
In this proposal, we present a project for a web application that can generate predic-
tions for housing prices in France. We wish to use Computer vision tools to help identify a pricing range mainly based on pictures from accomodations on sell. Examples of computer vision tools to test includes: VLFeat, Harris corner detector, SIFT, SURF, Fast corner detection, Kanade�Lucas�Tomasi feature trackers that would certainly be available in the openCV library in python programming langage. This approach permits an easy and fast way for potential house sellers to estimate a price for their accomodation before putting it on sell. It is highly probable that a picture only estimation will be less accurate than adding complementary pieces of information such as the geographic localisation of the accomodation, its presence either in a city or in contryside, the accomodation size and land around it. To select the model to estimate the estate price, we will test a range of machine learning algorithms such as: 
regression methods, Support Vector Regression (SVR), k-Nearest Neighbours (kNN), and Regression Tree$/$Random Forest Regression. 
To train the selected model, we will need to put together a dataset of accomodation pictures with a known price in France. We then could use it to fit local models with a variable geographic scope to test the importance of localisation in house pricing. As the training will occurs before deploying the web application and because most of the computation needed to estimate the pricing will happen on the server-side, we should be able to provide on the client-side a relative quick pricing estimation when a new accomodation needs to be evaluated on the web application. Finally, we will compare our approach to existing housing price prediction models to determine if our predictions are competitive enough.  
\newline
\noindent
\it Key words: Machine learning, Computer vision, Housing price, prediction\normalfont
\clearpage


\end{document}
