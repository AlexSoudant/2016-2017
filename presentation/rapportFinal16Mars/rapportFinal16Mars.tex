\documentclass[a4paper, 12pt, oneside]{report}
\raggedbottom
\usepackage{setspace}
\onehalfspacing
\newcommand{\HRule}{\rule{\linewidth}{0.5mm}}
\pdfminorversion=6
\usepackage{amssymb}
\usepackage{enumitem}
\usepackage[french]{babel}
\usepackage{wasysym}
\usepackage[a4paper,bindingoffset=0in,left=1in,right=1in,top=1in,bottom=1in,footskip=.25in]{geometry}
\usepackage[ansinew]{inputenc}
\usepackage{color}% colour for table cells
\usepackage{multirow}
\definecolor{light-gray}{gray}{0.6}
\usepackage{rotating}
\usepackage{booktabs}
\usepackage{sidecap}
\usepackage{caption}
    \captionsetup{font=small,labelfont=bf}
\usepackage{floatrow}
\usepackage{array,multirow}
\floatsetup[table]{capposition=top}
\parindent 0pt	

\usepackage{hyperref}

\title{\vspace{-15mm}\fontsize{24pt}{10pt}\selectfont\textbf{Machine Learning For House Pricing Prediction in Loire Atlantique Region}} % Article title

\author{
\large
\textsc{Alex Soudant}
\\[2mm] 
\normalsize Ynov Ing�sup M1,\\
\normalsize 20 Boulevard G�n�ral de Gaulle, 44200 Nantes \\
\\[2mm] 
\normalsize Correspondence: Alex Soudant. E-mail: \href{mailto:alex.soudant@ynov.com}{alex.soudant@ynov.com} % Your email address
\vspace{-5mm}
}

\date{}

\vspace{1cm}

\begin{document}

\maketitle

\vspace*{5cm}
\textbf{Abstract}

\noindent
\normalfont 
I present here a project to generate predictions for housing prices in Loire Atlantique Region (France) based on pictures analysis from sale adverts. I used the openCV librairy for computer vision to prepare pictures followed by the use of Tensorflow library for deep learning analysis. The goal is to produce an easy and fast way for potential house buyers$/$sellers to estimate a fair price for properties before contacting real estate agents. Results show that predictions based solely on pictures are not accurate enough to predict prices of properties on sale. I tested different approaches to improve prediction accuracy that will be detailed further in this report. The best set up so far was to build a neural network with both pictures and house features (i.e. number of rooms, house surface, land surface and number of pictures) as parallel neurones on the same layer then merge the resulting tensors in an output layer. To gather pictures and features necessary to this analysis, a scrapper was built using Ember framework and ran over a house sale agency website. About 600 adverts were obtained for Nantes city which represent over 3000 pictures. Angers and Brest city were also scraped but not used in training or validation for the model.  
\newline
\noindent
\it Key words: Machine learning, Computer vision, Housing price, prediction\normalfont
\clearpage


\textbf{Introduction}
\newline
\noindent
Housing is an important market for business
purposes. Nowadays, potential buyers$/$sellers for properties have to
manually visit a high number of housing agents websites to be able to estimate a range of price. This is a complex and time consuming process and the information obtained may not be accurate. It is also known that the difference in prices between websites for the same property can be quite high and may cause a lack of understanding of the real market prices. Hence I propose to develop a machine learning tool to assist individuals into finding a more understandable price from real estate.
\newline

\textbf{Data Obtention}
\newline
\noindent
To our knowledege, at the time of proposal writing, there was no public dataset available already published to fulfill such a study on the determination of housing price based on property prictures. This is due to the fact that sell prices are private in France so we cannot use recent archives of estate pricing. Therefore, I built a javscript scraper using Ember framework to help building a decent dataset to perform analysis. 
\newline

\textbf{Data Preparation and Cleaning}
\newline
\noindent
After running the scraper to gather pictures from the website into different advert folders in the hard drive, I noticed that some pictures were corrupted. The main consequence in such pictures was that the resolution was pretty poor and uneven thoughout the picture. Therefore, it can be expected that these pictures contain less information than a correctely downloaded picture which can greatly affect the analysis when training the model to predict prices. A easy way to get rid of these images is to read the size of the picture by using opencv library and deleting every picture below a threshold of 9000 octets. 
\noindent
To produce tensors based on image information, the original pictures need to be squared and of a size that will not impair too much computer performance. Therefore, all pictures were fisrt cropped to take only a central square part for analysis and then resized to obtain a resolution of 58 pixels for width and height. As I did not want to introduce color analysis for price estimation, all pictures were also converted to a gray scale which also help reduce picture size and so training speed.
\noindent
A last technique used on pictures is to artificially increase their numbers by swapping some pixels between each other and creating different pictures for a computer vision point of vue. To try this technique, I created three copies from the first image of each advert. The first copy have 58 pixels swapped randomly over the whole picture, the second copy is flipped horizontally and finally the last copy have both the 58 pixels randomly shuffled and all pixels flipped over.
\noindent
Features of the houses on sale, extracted from the online adverts, were saved into one json file for each advert. Therefore, the first quality check was to assert if all advert folders had a json file and delete folders that did not contained one. Once, I was sure to have in each advert folder, a set of pictures and a json file, the features for every advert were regrouped into a csv file in a table format. Features and pictures were then paired into pickle files following tensorflow tutorial recommandations. After having each picture paired with features, I then checked for prices and house surface equal to zero and removed these data lines before analysis. Adverts with no price indicated seem to match properties that need to be directly negociated by contacting the real estate agency. When no house surface is indicated, a rapid check on the real estate website shows that some adverts concern land properties with no built house on it.     
\noindent
Due to the fact that the scraper gather adverts from page 1 of the website to page 30, I expected that adverts from the beginning of the dataset could be of better quality (i.e. attrative prices, more pictures or more attractive features). Therefore, all adverts were first shuffled before segregating them into training and validation sets. The training dataset represents around 1900 paired pictures-features which leaves about 700 paired pictures-features for validation. As I consider these numbers to be still quite low to train deep learning algorithms, no test dataset was built to keep more data for training and validation. 
\noindent
Finally, all features were normalized to match a scale between -1 and 1. This is an essential transformation to allow the chosen optimizer to reach a minimum loss when running a graph session.  
\newline

\textbf{Building the Graph}
\newline
\noindent
Building a computational graph for tensorflow is one of the key step to obtain good results. It is also a particulary difficult task and needs to compute sessions many times to observe what layouts are better than others to obtain good validation accuracy. In that purpose, I tried a range of different combinations and evaluated their impact on results. However, it will be noticeable that the tested approaches are far from being a complete overview of potential techniques to improve the validation score and that my calculation of accuracy is not adequate to observe small size improvements over runs with a large amount of steps.     
\newline
\noindent
One way to potentially improve validation is to train a logistic model instead of a linear model. This can be done by creating classes of prices regrouping many adverts within a similar range of prices. Therefore, we also greatly reduce the number of prediction labels needed to run the graph. To perform this technique, I created ten classes of prices based on quartiles segregation so each classe has a similar number of adverts in it. This particularity permits that the model is trained on the same amount of information regardless of the price classes. 
\noindent
Another well known technique in tensorflow is to increase the depth of the graph by introducing hidden layers with RELU activation in between inputs and outputs. This permits the model to detect finer information as the pictures will go through each layer.
\noindent
Lastly, I tried to implement parallel neurons on the same hidden layer that will only process either pictures or house features. The resulting tensors are then merged before going through the output layer.
\newline

\textbf{Tuning the Graph}
\newline
\noindent
Tuning the model to perform better is another hot topic in deep learning. In the range of techniques that I could use, I triend to change from gradient descent optimizer to other types of optimizers. At the end, I kept adgrad optimizer to carry on analysing but the decrease in loss was not significant enough to justify such a choice.
\noindent
Changing the size of the hidden layer is also a possible way of improvement but it is also known to considerably increase running sessions times when it becomes bigger. Therefore, only small scale changes were tried which did not led to significant improvement. 
\noindent
A technique known as the learning rate decay was also tested which involve reducing progressively the learning rate when the loss stay stable for long enough that a small decrease in learning rate can permit an improvement. However, this also means that the loss will decrease slower and slower which increases run times. With a very unstable training accuracy, I preferred to set quite a low learning rate from the beginning. 
\noindent
To decrease run times, a good trick is to feed the graph with mini-batches instead of the full training set. However, the integrality of the validation set is kept when evaluating the model score. The size of the mini-batches can influence the score and needs to be assessed. For my computation needs, the mini-batches were set to a size of 40 which kept similar results than without the batch approach.
\newline


\textbf{Results}
\newline
\noindent


\textbf{Conclusion}
\newline
\noindent

\textbf{Going Further}
\newline
\noindent
I can think of two axis of improvement: Try convolutions over the pictures to evaluate if more information can be extracted to predict prices. The second axis is to build a better graph with a more complex architecture of hidden layers. More data et better computation run time by using gpu instead of cpu calculation would also be a great improvement.


\end{document}
